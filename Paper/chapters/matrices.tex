\setchapterpreamble[u]{\margintoc}
\chapter{Matrici}
\labch{matrices}

\section{Introduzione}

\paragraph{Struttura del codice}

Da questo capitolo in poi, il codice sorgente utilizzera come linguaggio primario
C++. La librerie necessarie prima di proseguire sono le seguenti:

\begin{itemize}
    \item \texttt{tensor.hpp} versione modificata di \texttt{matrix.h} disponibile
        su e-learning: l'header e' stato generalizzato per funzionare sia come
        vettori sia come matrici rendendo le operazioni compatibili fra i due e 
        facilitando il successivo svolgimento degli esercizi.
    \item \texttt{tensor\_utils.hpp} contentente varie funzionalita' utili e contente
        gli argomenti creati per ogni esercizio
    
\end{itemize}

\section{Esercizi}

\subsection{Inversione di matrici triangolari}

\subsection{Eliminazione di Gauss}

\subsection{Decomposizione LU}

