\setchapterpreamble[u]{\margintoc}
\chapter{Numeri}
\labch{numbers}

\section{Rappresentazione}

La rappresentazione numerica a cui il calcolo scientifico fa riferimento principalmente
è quella dei numeri reali; nell’ambito informatico tale rappresentazione utilizza
il concetto di numeri a virgola mobile come standard: i numeri reali vengono
rappresentati attraverso una notazione scientifica in base due tramite la seguente formula:

$$
	(-1)^S \left( 1 + \sum_n M_n 2^{-n} \right) \cdot 2^{E}
$$

Dove:
\begin{description}
	\item $S$ è il valore booleano per il \textbf{segno}
	\item $M$ è la parte decimale detta \textbf{mantissa}

	\item $E = e - d$ è l'\textbf{esponente} con $d$ (offset), $e$ (esponente dopo offset)
\end{description}

\section{Esercizi}

\subsection{Precisione}

\paragraph{Nozioni teoriche}

\subparagraph{Definizione} La \textit{precisione di macchina} (o \textit{$\epsilon$ di macchina})
è la differenza tra 1 e il numero successivo rappresentabile dato il numero di bit
richiesti, esso sarà dunque:

$$
	\epsilon = 2^{-M}
$$

Nello standard dei numeri a virgola mobile (IEE 754) si studiano principalmente
due sottoclassi di numeri i cui nominativi nei linguaggi C-like sono:

\begin{description}
	\item[float] numero a singola precisione (32 bit di memoria):
		\begin{itemize}
			\item $M$: 23 bit
			\item $E$: 8 bit
			\item Valore massimo: $3.40 \cdot 10^{38}$
			\item $\epsilon$: $\sim 10^{-7}$
		\end{itemize}
	\item[double] numero a doppia precisione (64 bit di memoria):
		\begin{itemize}
			\item $M$: 52 bit
			\item $E$: 11 bit
			\item Valore massimo: $1.8 \cdot 10^{308}$
			\item $\epsilon$: $\sim 10^{-16}$
		\end{itemize}

\end{description}

\paragraph{Richiesta} Scriverete un programma C che esegua le seguenti operazioni:

\begin{lstlisting}
    define f in single precision = 1.2e34
    for loop with 24 cycles:
        f *= 2
        print f in scientific notation
    repeat for d in double precision, starting from 1.2e304
    define d in double precision = 1e-13
    for loop with 24 cycles:
        d /= 2
        print d and 1+d in scientific notation
    repeat for single precision
\end{lstlisting}

Esaminare il range minimo e massimo e il ruolo dell'errore di macchina.

\paragraph{Implementazione e osservazioni}

\subparagraph{File necessari} sorgente: \texttt{number\_precision.c}, dati: \texttt{number\_precision.dat}

Il codice sorgente scritto utilizza funzionalità base del linguaggio C. \sidenote{
	Per formattare il codice secondo la richiesta del problema si usi la
	definizione \texttt{EXERCISE\_FORMAT}, altrimenti verrà utilizzata una formattazione
	più compatta per leggere in maniera più diretta i dati, si consiglia di utilizzare
	quest'ultima per comprendere l'analisi sottostante}

\paragraph{Analisi e conclusioni}

Dai dati ottenuti si possono notare in maniera esaustiva varie proprietà dei
numeri a virgola mobile:

\begin{enumerate}
	\item Esiste un \textit{valore massimo} sia per singola ($\sim 3 \cdot 10^{38}$) sia
	      per doppia precisione ($\sim 2 \cdot 10^{308}$), superato esso viene mostrato un
	      valore esatto \textit{inf} definito dallo standard descritto in precedenza;

	\item I numeri hanno un \textit{errore macchina} dettato dalla capienza di memoria della mantissa;

	\item Come mostrerà più precisamente la prossima sezione, l’errore viene
	      \textit{propagato} nella somma:

	      \begin{itemize}
		      \item $1 + f_{mult}$ perde completamente l’informazione su $f_{mult}$

		      \item $1 + d_{mult}$ la conserva soltanto per le prime iterazioni;
	      \end{itemize}
\end{enumerate}
\subsection{Propagazione degli errori}
\paragraph{Nozioni teoriche}

E' immediato notare come i numeri a virgola mobile possano essere rappresentati come
variaili casuali con errore associato, derivante dalla precisione di macchina.

Prendiamo in esame una funzione $f(x, y)$ dove $x, y$ sono variabili casuali indipendenti
con rispettivo errore $\sigma_x, \sigma_y$, allora l'errore su f sarà:

$$
	\sigma_f^2 = {\left( \frac{\partial f}{\partial x} \right)^2 \sigma_x^2 + \left( \frac{\partial f}{\partial y} \right)^2 \sigma_y^2}
$$

Assumendo ora $f = x + y$ otteniamo:

$$
	\sigma_f^2 = {\sigma_x^2 + \sigma_y^2}
$$

Notiamo immediatamente quindi che se $x \gg y $ allora $\sigma_f \approx \sigma_x$ quindi
si perde l'informazione su $y$ nella somma.


\paragraph{Richiesta} 

Si scriva in C un programma che esegua le seguenti operazioni:

\begin{lstlisting}
    calculate (0.7 + 0.1) + 0.3 and print 16 digits
    calculate 0.7 + (0.1 + 0.3) and print 16 digits
    define xt = 1.e20; yt = -1.e20; zt = 1
    calculate (xt + yt) + zt
    calculate xt + (yt + zt)
\end{lstlisting}

Esaminare la non-associatività dell'addizione e il ruolo degli errori di arrotondamento.



\paragraph{Implementazione e osservazioni}

\subparagraph{File necessari} sorgente: \texttt{error\_propagation.c}

In base alle richieste l'output è il seguente:

\begin{enumerate}

	\item $(0.7 + 0.1) + 0.3 =_? 0.7 + (0.1 + 0.3)$:
	      $$\texttt{Output: 1.1000000238418579, 1.1000000238418579}$$

	      La somma risulta associativa.

	\item $[10^{20} + (-10^{20})] + 1 =_? 10^{20} + [(-10^{20}) + 1]$:
	      $$\texttt{Output: 1.0000000000000000, 0.0000000000000000}$$

	      La somma risulta non associativa.
\end{enumerate}

\label{sec:propagation}
\paragraph{Analisi}

Utilizzando le formule discusse si può studiare la propagazione dell’errore nella somma.
In essa la propagazione dipende dall’errore assoluto dei singoli addendi.
Assumendo numeri a singola precisione e ricordando che $ \sigma_x \approx \epsilon \sim 10^{-7}$,
si ottengono i seguenti casi:

\begin{enumerate}

	\item Per i valori 0.7, 0.1, 0.3 l’ordine di grandezza è lo stesso, quindi,
	      tutti i valori possegono un errore assoluto $\Delta x \sim 10^{-8}$; propagando
	      l’errore nella somma si ottiene dunque $\Delta_{output} \sim 3 \cdot \Delta x$ in accordo con
	      i risultati.

	\item Il risultato è describile come un caso limite nell’errore di propagazione rispetto
	      alla singola precisione, infatti, $10^{20}$ avrà un errore assoluto di $\sim 10^{13}$
	      mentre 1 di $10^{-7}$!

	      La spiegazione dell'output ottenuto, dunque, si basa sulla differenza
	      tra ordini di grandezza dei diversi addendi:
	      \begin{itemize}
		      \item Nel termine a sinistra
		            vengono sommati prima numeri con errore assoluto paragonabile.
		            Si ottiene quindi $\sim 0$ che sarà poi sommato con un numero
		            avente errore assoluto simile a 1.

		      \item
		            Nel termine a destra, invece, si sommano due valori con venti
		            ordini di grandezza di differenza: l’errore assoluto di
		            $10^{20}$ prevale e si perde qualsiasi informazione nella somma per termini:
		            $$x \ll 10^{20} \Rightarrow x + 10^{20} \sim 10^{20}$$
		            Segue che $\mathit{1}$ sarà ignorato nella somma a destra.
	      \end{itemize}

\end{enumerate}

\paragraph{Conclusioni}
Nel manipolare numeri in un calcolatore l’operazione eseguita, la precisione e
la differenza in ordine di grandezza dei numeri partecipanti devono essere tenuti
sempre in considerazione specialmente nelle addizioni.

