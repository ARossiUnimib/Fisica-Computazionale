\setchapterpreamble[u]{\margintoc}
\chapter{Numeri}
\labch{numbers}

\section{Rappresentazione}

La rappresentazione numerica a cui il calcolo scientifico si riferisce principalmente
è quella dei numeri reali; nell’ambito informatico tale rappresentazione utilizza
il concetto di numeri a virgola mobile come standard: i numeri reali vengono
rappresentati attraverso una notazione scientifica in base due tramite la seguente formula:

$$
	(-1)^S \left( 1 + \sum_n M_n 2^{-n} \right) \cdot 2^{E}
$$

Dove:
\begin{description}
	\item $S$ è il valore booleano per il \textbf{segno}
	\item $M$ è la parte decimale detta \textbf{mantissa}

	\item $E = e - d$ è l'\textbf{esponente} con $d$ (offset), $e$ (esponente dopo offset)
\end{description}

% inserire formula

\section{Precisione: Esercizio 1}


\paragraph{Nozioni teoriche} \mbox{}\\

\textit{Definizione}: La \textbf{precisione macchina} (o \textit{$\epsilon$ di macchina})
è la differenza tra 1 e il numero successivo rappresentabile dato il numero di bit
richiesti, esso sarà dunque:

$$
	\epsilon = 2^{-M}
$$


Nello standard dei numeri a virgola mobile (IEE 754) si studiano principalmente
due sottoclassi di numeri i cui nominativi nei linguaggi C-like è:

\begin{description}
	\item[float] numero a singola precisione (32 bit di memoria):
		\begin{itemize}
			\item $M$: 23 bit
			\item $E$: 8 bit
			\item Valore massimo: $3.40 \cdot 10^{38}$
			\item $\epsilon$: $\sim 10^{-7}$
		\end{itemize}
	\item[double] numero a doppia precisione (64 bit di memoria):
		\begin{itemize}
			\item $M$: 52 bit
			\item $E$: 11 bit
			\item Valore massimo: $1.8 \cdot 10^{308}$
			\item $\epsilon$: $\sim 10^{-16}$
		\end{itemize}

\end{description}

\paragraph{Implementazione e osservazioni} \mbox{}\\
\\
\textit{File necessari}: sorgente: \texttt{number\_precision.c}, dati \textt{number\_precision.dat}
\\

% TODO: aggiungere grafico?

\paragraph{Analisi e conclusioni}

Nel
primo esercizio si possono notare in maniera esaustiva varie propriet`a dei
numeri a virgola mobile:
1. Esiste un valore massimo sia per singola (∼ 3·1038) sia per doppia pre-
cisione (∼ 2 · 10308), superato esso viene mostrato un valore esatto inf
definito dallo standard descritto in precedenza;
2. I numeri hanno un errore macchina dettato dalla capienza di memo-
ria della mantissa;
3. Come mostrer`a pi`u precisamente la prossima sezione, l’errore viene
propagato nella somma:
• 1 + fmult perde completamente l’informazione su fmult
• 1 + dmult la conserva soltanto per le prime iterazioni;

\section{Propagazione degli errori: Esercizio 2}

\paragraph{Nozioni teoriche}

\paragraph{Implementazione e osservazioni}\mbox{}\\
\\
\textit{File necessari}: sorgente: \texttt{error\_propagation.c}
\\

\begin{enumerate}

	\item $(0.7 + 0.1) + 0.3 =_? 0.7 + (0.1 + 0.3)$:
	      $$\texttt{Output: 1.1000000238418579, 1.1000000238418579}$$

	      La somma risulta associativa.

	\item $[10^{20} + (−10^{20})] + 1 =_? 10^{20} + [(−10^{20}) + 1]$:
	      $$\texttt{Output: 1.0000000000000000, 0.0000000000000000}$$

	      La somma risulta non associativa.
\end{enumerate}

\paragraph{Analisi}

Utilizzando le formule discusse si può studiare la propagazione dell’errore nella somma.
In essa la propagazione dipende dall’errore assoluto dei singoli addendi.
Assumendo numeri a singola precisione e ricordando che $ \epsilon \sim 10^{−7}$,
si studiano i casi ottenuti:

\begin{enumerate}

	\item Per i valori 0.7, 0.1, 0.3 l’ordine di grandeza è lo stesso, quindi,
	      tutti i valori possegono un errore assoluto $\Delta x \sim 10^{-8}$, propagando
	      l’errore nella somma si ottiene dunque $\Delta_{output} \sim 3 \cdot \Delta x$ in accordo con
	      i risultati.

	\item Esso è describile come un caso limite nell’errore di propagazione rispetto
	      alla singola precisione, infatti, 1020 avr`a un errore assoluto di ∼ 1013
	      mentre 1 di 10−7!

	      E’ quindi ragionevole spiegare
	      l’output in termini di ordini di grandezza. Nel termine di sinistra ven-
	      gono sommati prima numeri con errore assoluto paragonabile si otterr`a
	      dunque ∼ 0 che sar`a poi sommato con un numero avente un errore
	      assoluto simile a 1.
	      Nel termine a destra, invece, si sommano due
	      valori con venti ordini di grandezza di differenza: l’errore assoluto di
	      1020 prevale e si perde qualsiasi informazione nella somma per termini
	      x << 1020 ⇒ x + 1020 ∼ 1020, ’1’ `e quindi ignorato nella somma a
	      destra.
\end{enumerate}

\paragraph{Conclusioni}
Nel manipolare numeri in un calcolatore l’operazione eseguita, la precisione e
la differenza in ordine di grandezza dei numeri partecipanti devono essere tenuti
sempre in considerazione specialmente nelle addizioni.

