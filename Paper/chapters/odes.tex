\setchapterpreamble[u]{\margintoc}
\chapter{ODE: Equazioni differenziali ordinarie}
\labch{odes}

\section{Esercizi}

\subsection{Oscillatore approssimato}

\paragraph{Richiesta}

Risolvere l'equazione differenziale dell'oscillatore armonico con i seguenti valori iniziali:

\begin{equation}
\ddot{\theta}(t) = -\theta(t)
\end{equation}
con $\theta(0) = 0$ e $\dot{\theta}(0) = 1$.

\begin{enumerate}
    \item Utilizzare il metodo di Eulero, il metodo di Runge-Kutta del secondo ordine (RK2) e il metodo di Runge-Kutta del quarto ordine (RK4).

    \item Ottenere la soluzione analitica $\theta(t)$ e confrontarla con la soluzione numerica $\eta(t;h)$ ottenuta con i tre metodi. Studiare l'errore:
   \[
   e(t,h) = \eta(t,h) - \theta(t)
   \]
   in funzione del passo $h$ e verificare che i metodi abbiano l'ordine atteso.

\end{enumerate}

\subsection{Oscillatore reale}

\paragraph{Richiesta}

Considerare l'equazione differenziale del pendolo senza l'approssimazione delle piccole oscillazioni:

\begin{equation}
\ddot{\theta}(t) = -\sin{\theta(t)}
\end{equation}
con $\theta(0) = 0$ e $\dot{\theta}(0) = 1$.

\begin{enumerate}
\item Risolvere numericamente l'equazione differenziale e tracciare il grafico di $\theta(t)$ e $\dot{\theta}(t)$.
\item Ripetere l'esercizio includendo un termine di attrito:
   \[
   \ddot{\theta}(t) = -\sin{\theta(t)} - \gamma \dot{\theta}(t)
   \]
\item e un termine forzante: 
   \[
   \ddot{\theta}(t) = -\sin{\theta(t)} - \gamma \dot{\theta}(t) + A \sin{\omega t}
   \]
   con $\gamma \in (0, 2)$ e $\omega \in (0, 2)$.

\end{enumerate}

\subsection{Attrattore di Lorenz}

\paragraph{Richiesta}

Studiare il sistema di ODE:

\begin{equation}
\begin{cases}
\dot{x}(t) = 10(y(t) - x(t)) \\
\dot{y}(t) = 28x(t) - y(t) - xz(t) \\
\dot{z}(t) = -8/3 z(t) + xy(t)
\end{cases}
\end{equation}

\begin{enumerate}
\item Utilizzare il metodo di Eulero, il metodo di Runge-Kutta del secondo ordine (RK2) e il metodo di Runge-Kutta del quarto ordine (RK4).
\item Tracciare il grafico di $x(t)$, $y(t)$ e $z(t)$.
\end{enumerate}

\subsection{Sistema a tre corpi}

\paragraph{Richiesta}

Studiare il sistema gravitazionale che obbedisce alle equazioni del moto:

\begin{equation}
\ddot{\vec{x}}_i = \sum_{j \neq i} m_j \; \frac{ \vec{x}_j - \vec{x}_i}{|\vec{x}_j - \vec{x}_i|^3}
\end{equation}

Nel caso di tre masse puntiformi, $i = 1, 2, 3$, con i seguenti parametri e condizioni iniziali:

\begin{itemize}
    \item 
        \quad\parbox{\linewidth}{
            \begin{math}
            m_1 = m_2 = m_3 = 1 \\ 
            \vec{x}_1 = (1, 0, 0), \quad \dot{\vec{x}}_1 = (0, 0.15, -0.15) \\  
            \vec{x}_1 = (-1, 0, 0), \quad \dot{\vec{x}}_1 = (0, -0.15, 0.15) \\ 
            \vec{x}_1 = (0, 0, 0), \quad \dot{\vec{x}}_1 = (0, 0, 0)  
            \end{math}
            \\
        }
    \item 
        \quad\parbox{\linewidth}{
            \begin{math}
            m_1 = 1.6, m_2 = m_3 = 0.4 \\ 
            \vec{x}_1 = (1, 0, 0), \quad \dot{\vec{x}}_1 = (0, 0.4, 0) \\  
            \vec{x}_1 = (-1, 0, 0), \quad \dot{\vec{x}}_1 = (0, -0.8, 0.7) \\ 
            \vec{x}_1 = (0, 0, 0), \quad \dot{\vec{x}}_1 = (0, -0.8, -0.7)  
            \end{math}
        }
\end{itemize}

\begin{enumerate}
\item Risolvere numericamente il sistema di equazioni.
\item Tracciare il grafico dell'energia totale del sistema in funzione del tempo.
\end{enumerate}
