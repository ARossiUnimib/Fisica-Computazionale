\setchapterpreamble[u]{\margintoc}
\chapter{Intepolazione}
\labch{interp}


\section{Implementazione preliminaria}

Prima di procedere è necessario specificare varie funzioni/classi appositamente create
per la scrittura del codice e una più facile implementazione degli algoritmi.

\paragraph{Classe \texttt{FunctionData}}

La classe \texttt{FunctionData} ha lo scopo di conservare i dati numerici ottenuti
dalla computazione delle varie funzioni (in questo modulo principalmente polinomi) \sidenote{Nell'implementazione della classe
	può essere trovata anche un'implementazione apposita di un iterator per
	facilitare l'utilizzo dei valori quando ciclati.}.

Essa ha come membri due vettori di tipo \texttt{std::vector} che conservano le
informazioni di $x$ e $f(x)$.

\paragraph{Classe \texttt{Range}}

La classe \texttt{Range} ha invece lo scopo di rappresentare una successione di
punti $\left\{a_i\right\}_{i \in (a, b) \subset \mathbb{R}}$ (la rappresentazione
prevede intervalli inclusivi ed esclusivi).

Scopo principale della classe sarà quello di generare nodi di distanza finita e
nodi di Chebyshev.

\textit{Ulteriori informazioni sulle due classi possono essere trovate nella
	documentazione dedicata negli appositi header} \texttt{function\_data.hpp} \textit{e}
\texttt{range.hpp}.

\section{Esercizi}

\subsection{Metodo diretto e polinomio di Newton}

\paragraph{Nozioni teoriche}

Il metodo più semplice per ottenere il polinomio di interpolazione è quello di ottenere

% TODO:

\paragraph{Implementazione metodo diretto}

\subparagraph{File necessari} \texttt{interpolation.hpp}

Come primo step si ottiene la matrice di Vandermonde dalla sua definizione:

\begin{lstlisting} [language=C++]

// "values" sono i valori {x_1, x_2, ...} da inserire 
template <typename T>
tensor::Tensor<T> VandermondeMatrix(std::vector<T> const &values) {
  auto mat = tensor::Tensor<T>::SMatrix(values.size());

  for (int i = 0; i < values.size(); i++) {
    for (int j = 0; j < values.size(); j++) {
      mat(i, j) = pow(values[i], j);
    }
  }

  return mat;
}
    
\end{lstlisting}

Successivamente si risolve il sistema lineare utilizzando il metodo di Gauss
(invertire la matrice necessiterebbe di un ulteriore eliminazione di Gauss ed
ulteriore allocamento di memoria).

\begin{lstlisting} [language=C++]
template <typename T>
std::vector<T> DirectCoefficients(func::FunctionData<T> const &f) {
  auto f_tensor = tensor::Tensor<T>::FromData(f.F());

  tensor::Tensor<T> vande_matrix = VandermondeMatrix(f.X());
  tensor::Tensor<T> values =
      tensor::GaussianElimination(vande_matrix, f_tensor);

  // Trasforma l'oggetto tensore in std::vector
  return values.RawData();
}

\end{lstlisting}

\paragraph{Implementazione metodo di Newton}

Analogamente al metodo diretto calcoliamo i coefficienti per il polinomio di
newton seguendo la formula citata precedentemente otteniamo.

\begin{lstlisting} [language=C++]
template <typename T>
std::vector<T> NewtonCoefficients(func::FunctionData<T> const &f) {
  int N = f.Size();

  std::vector<T> a(N);

  auto A = tensor::Tensor<double>::SMatrix(N);

  // Rempiamo la prima colonna con i valori della funzione
  for (int i = 0; i < N; i++) {
    A(i, 0) = f.F(i);
  }

  // Calcoliamo le differenze divise
  for (int j = 1; j < N; j++) {
    for (int i = 0; i < N - j; i++) {
      A(i, j) = (A(i + 1, j - 1) - A(i, j - 1)) / (f.X(i + j) - f.X(i));
    }
  }

  // Estrapoliamo i coefficienti
  for (int i = 0; i < N; i++) {
    a[i] = A(0, i);
  }

  return a;
}
\end{lstlisting}

\subparagraph{Considerazioni} L'algoritmo è direttamente implementato rispetto a
lla logica che abbiamo considerato per calcolare i coefficienti di Newton.
Uno svantaggio di questo algoritmo sta nell utilizzo dell'oggetto Tensor, infatti
il funzionamento interno della classe traduce una matrice 2x2 in una vettore
contiguo: questo permette un ottimizzazzione della cache della cpu rispetto ad un
vettore di vettori, ma nel caso utilizzato la matrice è solo occupata a metà, 
lasciando inizializzati a 0 molti valori non utilizzati nella computazione dei
coefficient.
Un modo per evitare ciò sarebbe per esempio quello di implementare un oggetto ad hoc
per il problema, ai fini concettuali, però, l'implementazione sarebbe la medesima.

\paragraph{Analisi dei risultati}

Dalla implementazione dei metodi sopra citati e dai dati forniti dal problema
otteniamo le seguenti interpolazioni:


\subsection{Funzione di Runge}

\subsection{Esercizio 3?}
