\setchapterpreamble[u]{\margintoc}
\chapter{Introduzione}
\labch{intro}

\section{Struttura dell'elaborato}

Nel seguente elaborato verrà illustrata l’analisi degli argomenti proposti nel
corso di Fisica Computazionale.

Di seguito sono riportate le informazioni richieste per una fruizione com-
pleta del documento:

\begin{description}
    \item[Codice sorgente e dati] Il codice sorgente ed i dati analizzati nei vari esercizi sono reperibili al
seguente repository Github nelle cartelle dei capitoli omonimi.

• La lingua utilizzata nel codice sorgente sar`a l’inglese per avere una
maggiore coesione sintattica con i linguaggi di programmazione utiliz-
zati.
• Verranno ripetute, sopratutto nella parte introduttiva dei vari capi-
toli, i punti chiave recuperati dalle risorse disponibili sull’e-learning del
corso. Esse saranno riassuntive favorendo le precisazioni e lo studio
degli esercizi per ottenere un quadro completo dei vari argomenti.

Many modern printed textbooks have adopted a layout with prominent 
margins where small figures, tables, remarks and just about everything 
else can be displayed. Arguably, this layout helps to organise the 
	discussion by separating the main text from the ancillary material, 
	which at the same time is very close to the point in the text where 
	it is referenced.
\end{description}

% \sidenote{This also means that 

% \labsec{does}

% \Class{kaobook}  
% \vrefsec{doesnot}). 
% \labsec{doesnot}

% \marginnote[2mm]{to do!}
